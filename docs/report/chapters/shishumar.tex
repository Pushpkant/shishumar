%shishumar

\chapter*{Shishumar}

The {\bf Platanista Gangetica} or {\bf Ganges River Dolphin} are fresh water dolphin, lives in the Ganges and related rivers of South Asia\cite{wwf_ganges_river_dolphins}. They are commonly known as {\bf susu, soos, pani suar, shushuk, hihu} in north India. The name {\bf Shishumar} is from medieval Sanskrit name of Ganges Dolphins.

The Shishumar has been listed as an Endangered species on the \gls{iucn_red} since 1996. Prime Minister of India announced the {\bf Project Dolphin} during 74th independence day celebration on 15 August 2020\cite{pib_pmo_74_independence_day}. India's Dolphins are at risk of extinction due to canal systems, construction of waterways, unchecked fishing activities etc\cite{joshi_2023}. Project Dolphin involves status monitoring and their potential threats, in order to develop and implement a conservation action plan for protecting dolphins and aquatic habitat. As a part of project October 5th has been designated as ``National Dolphin Day'' by environment ministry.